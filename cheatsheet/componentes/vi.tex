\bcode{vi} es un editor de texto presente en todos los sistemas UNIX, incluido Linux. Para entrar, basta hacer: \bcode{vi \emph{ficheroDeTexto}}. Una vez dentro tiene dos modos principales de operación: inserción y comando. Al modo inserción se entra pulsando la tecla \code{i} y se sale (de nuevo al modo comando) dando \code{ESC}. En modo comando hay varias acciones típicas:\\
  \bcode{:wq} Guarda el fichero y sale del editor.\\
  \bcode{:q!} Sale del editor sin guardar el fichero.\\
  \bcode{/palabra} Busca en el fichero.\\
  \bcode{i} Entra en modo inserción.\\
  \bcode{o} Entra en modo inserción en la línea siguiente.
%  \bcode{yyp} Duplica la linea donde está el cursor (\code{yy} la copia y \code{p} la pega debajo de éste).
